%arara : pdflatex
\documentclass[12pt]{article}

\usepackage{../../TP0/style}

\begin{document}
\def\reportnumber{2}
\def\reporttitle{Algorithmes de Complexités temporelles linéaire O(n) et racine carrée O($\sqrt{n}$).}
%----------------------------------------------------------------------------------------
%	TITLE PAGE
%----------------------------------------------------------------------------------------


\begin{titlepage} % Suppresses displaying the page number on the title page and the subsequent page counts as page 1
	\newcommand{\HRule}{\rule{\linewidth}{0.5mm}} % Defines a new command for horizontal lines, change thickness here
	
	\center % Centre everything on the page
	
	%------------------------------------------------
	%	Headings
	%------------------------------------------------
	
	\baselineskip=2\baselineskip 
	\textsc{\LARGE Université des Sciences et de la Technologie Houari Boumediene}%\\[1cm] % Main heading such as the name of your university/college

	%------------------------------------------------
	%	Logo
	%------------------------------------------------
	
	%\vfill\vfill
	\vfill
	\includegraphics[width=0.3\textwidth]{../style/USTHB_Logo.png}\\[1cm] % Include a department/university logo - this will require the graphicx package
	 
	%----------------------------------------------------------------------------------------
	
	\textsc{\Large Compilation}\\[0.5cm] % Major heading such as course name
	%\textsc{\large Minor Heading}\\[0.5cm] % Minor heading such as course title
	
	%------------------------------------------------
	%	Title
	%------------------------------------------------
	
	\HRule\\[0.4cm]
	\baselineskip=1.2\baselineskip 
	{\huge\bfseries Rapport de Projet \\
	 \reporttitle}\\[0.4cm] % Title of your document
	
	\HRule\\[1.5cm]
	
	%------------------------------------------------
	%	Author(s)
	%------------------------------------------------
	
	\begin{minipage}{0.4\textwidth}
		\begin{flushleft}
			\large
			\textit{Binôme: groupe 4}\\
			HOUACINE  \textsc{Naila Aziza} % Your name
			\\
			MOHAMMEDI \textsc{Haroune } % Your name
			
		\end{flushleft}
	\end{minipage}
	~
	\begin{minipage}{0.4\textwidth}
		\begin{flushright}
			\large
			\textit{Professeur}\\
			Mme. MEKAHLIA \textsc{Fatma Zohra} % Supervisor's name
		\end{flushright}
	\end{minipage}
	
	%------------------------------------------------
	%	Date
	%------------------------------------------------
	
	\vfill\vfill\vfill % Position the date 3/4 down the remaining page
	
	{\large\today} % Date, change the \today to a set date if you want to be precise
	
	
	\vfill % Push the date up 1/4 of the remaining page
	
\end{titlepage}


\section{Partie I: Algorithme 1 du test de la primalité.}

\subsection{Développement de l'algorithme qui permet de déterminer  si un nombre entier naturel n est premier (n$>$=2). }
Afin de vérifier si un nombre entier naturel "n" est premier ou pas nous allons tester s'il est divisible par un autre nombre entier naturel appartenant à l'intervalle $[2 - n]$.
Pour cela nous allons utiliser la fonction modulo qui donne le reste de la division de n par i, i variant de 2 jusqu'à n.


\begin{sql}

 Algorithme_nombre_premier1

 VAR
 i,N : entier;
 prem : booleen;
 
 DEBUT
 
	écrire("donner la valeur de N = ");
	lire(N);

	i=2;
	prem = vrai;

	tant que( i <= N-1 ET prem == vrai){

		si( N mod i == 0)
			alors 
				prem = faux;				
			sinon
				i = i + 1;
	}

	si(prem == 1)
    	alors
        	écrire("Le nombre saisi :",N,"est premier!");
    	sinon
        	écrire("Le nombre saisi :",N,"n'est pas premier! ");
 FIN. 
 
\end{sql}

\subsection{Complexité:}

\subsubsection{Calcule des complexités temporelles en notation exacte et/ou en notation asymptotique de Landau O (Grand O) de  cet  algorithme au meilleur cas, notée f1(n), et au pire cas, notée f2(n). }
Le calcule de la complexité exact de cet algorithme n'est pas possible car nous ne pouvons pas déterminer une forme générale représentant les nombres premiers.
 
\begin{enumerate}
	\item Calcule de la complexité au meilleur cas:
	\\
	Il s'agit du cas ou le nombre est égale à 2 donc la boucle n'est exécutée aucune fois, tel que :
	\\
	f1(n) = 1(=) + 1(=) + 4($<$= , - , == , et) 
	\\
	\color{blue}
	f1(n) = 6 (Opérations) $\Rightarrow$ f1(n) = O(1)
	\color{black}
	\\
	\item Calcule de la complexité au pire cas:
	\\
	Il s'agit du cas ou le nombre est premier c'est à dire le contenu de la boucle est exécuté (n-1)-2 + 1 fois = n-2 fois.
	(selon la règle : fin-debut+1)
	\\
	ce qui donne:
	\\
	f2(n) = 1(=) + 1(=) + 4($<$=,-,==,et)(n-1) + [2(mod,==)+2(+,=)](n-2) 
	\\
	\color{blue}
	f2(n) = 8n - 10 (Opérations) $\Rightarrow$ f2(n) = O(n)
	\color{black}
\end{enumerate}




\subsubsection{Calculer la complexité spatiale en notation exacte et/ou en notation asymptotique de Landau O (Grand O) de  cet  algorithme notée s(n).}
Il s'agit du nombre de case mémoire ou octets utilisés par le programme.
\\
Pour calculer la complexité spatiale de cet algorithme nous allons considérer les tailles mémoires des types en langage C; tel que nous aurons:
int/entier : 2 octets
\\
\begin{enumerate}
	\item En notation exacte:
	\\
Nous avons dans cet algorithme que trois (3) variables de type "entier"
\\
Ce qui nous fait :
\color{blue}
 3(2 octets) = 6 octets
\color{black}
\\
 
	\item En notation asymptotique:
	\\
	Le nombre de case mémoire est constant, donc on obtient:
	\\
	\color{blue}
	s(n) = O(1)
	\color{black}
	
	
\end{enumerate}



\subsection{Développement de programme correspondant avec le langage C.}


\begin{sql}
#include <stdlib.h>
#include <stdio.h>

int main()
{

	int i,N,prem;


	printf("donner la valeur de N = ");
	scanf("%d",&N);

	i=2;
	prem = 1;

	while( i <= N-1 && prem == 1){

		if( N%i == 0)
			prem = 0;
		else
			i = i + 1;
	}

	if(prem == 1)
    {
        printf("Le nombre saisi : %d est premier! \n",N);
    }
	else{
        printf("Le nombre saisi : %d n'est pas premier! \n",N);
	}


return 0;

}

\end{sql}



\subsection{Vérification par programme  si  les  nombres  n  donnés  dans  le tableau de l'énoncé (1.000.003, 2.000.003, …) sont premiers.}

Pour répondre à cette question, notre programme doit contenir une tableau tab[ ] des valeurs à tester, ainsi le programme de teste de primalité devra être exécuté pour chaque élément de ce tableau.

\subsubsection{Programme C à exécuter:}

\begin{sql}
#include <stdlib.h>
#include <stdio.h>
#include <time.h>

int main()
{
	long int i,j,prem;

	long int tab[]={1000003, 2000003, 4000037, 8000009, 16000057, 32000011, 64000031, 128000003, 256000001, 512000009,	1024000009,	2048000011};

for(j=0 ; j<12 ; j++)
{
	i=2;
	prem = 1;
	
	while( i <= tab[j]-1 && prem == 1){

		if( tab[j]%i == 0)
			prem = 0;
		else
			i = i + 1;
	}

	if(prem == 1)
    {
        printf("Le nombre saisie : %ld est premier! \n",tab[j]);
    }
	else{
        printf("Le nombre saisie : %ld n'est pas premier! \n",tab[j]);
	}
}

return 0;
}
\end{sql}

\subsubsection{Résultats de l'exécution du programme:}
\begin{sql}
Le nombre saisie : 1000003 est premier!
Le nombre saisie : 2000003 est premier!
Le nombre saisie : 4000037 est premier!
Le nombre saisie : 8000009 est premier!
Le nombre saisie : 16000057 est premier!
Le nombre saisie : 32000011 est premier!
Le nombre saisie : 64000031 est premier!
Le nombre saisie : 128000003 est premier!
Le nombre saisie : 256000001 est premier!
Le nombre saisie : 512000009 est premier!
Le nombre saisie : 1024000009 est premier!
Le nombre saisie : 2048000011 est premier!
\end{sql}

On remarque que tous les nombres données sont premiers!

\subsection{Mesure des temps d'exécution T pour les nombres n données.}

Pour mesurer le temps d'exécution du programme nous utilisons les fonctions de gestion du temps qui sont fournies dans la bibliothèque "time.h" .

\subsubsection{Programme C correspondant au calcule du temps d'exécution pour chaque valeur du tableau:}
\begin{sql}
#include <stdlib.h>
#include <stdio.h>
#include <time.h>

int main()
{
	long int i,j,prem;
	clock_t deb,fin;
	double total;

	long int tab[]={1000003, 2000003, 4000037, 8000009, 16000057, 32000011,	64000031, 128000003, 256000001,	512000009,	1024000009, 2048000011};

for(j=0 ; j<12 ; j++)
{
	deb = clock();
	
	i=2;
	prem = 1;

	while( i <= tab[j]-1 && prem == 1){

		if( tab[j]%i == 0)
			prem = 0;
		else
			i = i + 1;
	}

	fin = clock();

	if(prem == 1)
    {
        printf("Le nombre saisie : %ld est premier! \n",tab[j]);
    }
	else{
        printf("Le nombre saisie : %ld n'est pas premier! \n",tab[j]);
	}

	total = (double) (fin - deb)/CLOCKS_PER_SEC;
	printf("temps d'exécution = %lf \n",total);
}
return 0;
}
\end{sql}

\subsubsection{Résultat de l'exécution du programme:}
\begin{sql}
Le nombre saisie : 1000003 est premier!
temps d`exécution = 0.009000
Le nombre saisie : 2000003 est premier!
temps d`exécution = 0.009000
Le nombre saisie : 4000037 est premier!
temps d`exécution = 0.013000
Le nombre saisie : 8000009 est premier!
temps d`exécution = 0.026000
Le nombre saisie : 16000057 est premier!
temps d`exécution = 0.051000
Le nombre saisie : 32000011 est premier!
temps d`exécution = 0.102000
Le nombre saisie : 64000031 est premier!
temps d`exécution = 0.205000
Le nombre saisie : 128000003 est premier!
temps d`exécution = 0.411000
Le nombre saisie : 256000001 est premier!
temps d`exécution = 0.820000
Le nombre saisie : 512000009 est premier!
temps d`exécution = 1.642000
Le nombre saisie : 1024000009 est premier!
temps d`exécution = 3.289000
Le nombre saisie : 2048000011 est premier!
temps d`exécution = 6.576000
\end{sql}


\subsubsection{Remplissage du tableau:}
\color{blue}
\textrm{  }
\\
\\
\begin{tabular}{|p{3cm}||p{1.8cm}|p{1.8cm}|p{1.8cm}|p{1.8cm}|p{1.8cm}|p{1.8cm}|}
\hline
Valeur N : & 1000003 & 2000003 & 4000037 & 8000009 & 16000057  & 32000011\\
\hline
Temps d'exe : & 0.009000 & 0.009000 & 0.013000 & 0.026000 & 0.051000 & 0.102000 \\
\hline
\end{tabular}
\\
\\
\begin{tabular}{|p{3cm}||p{1.8cm}|p{1.8cm}|p{1.8cm}|p{1.8cm}|p{1.8cm}|p{1.8cm}|}
\hline
Valeur N : & 64000031 & 128000003 & 256000001 & 512000009 &  1024000009 & 2048000011\\
\hline
Temps d'exe : &  0.205000 & 0.411000 & 0.820000 & 1.642000 & 3.289000 & 6.576000 \\
\hline
\end{tabular}


\textrm{  }
\\
\color{black}



\subsection{Développement du programme de mesure du temps d'exécution du programme qui a en entrée les données de l'échantillon dans tab1 et en sortie les temps d'exécution dans tab2. }
\begin{sql}
#include <stdlib.h>
#include <stdio.h>
#include <time.h>

int main()
{
	long int i,j,prem;
	clock_t deb,fin;
	double total;

	long int tab1[]={1000003, 2000003,	4000037,	8000009,	16000057,	32000011,	64000031,
	128000003,	256000001,	512000009,	1024000009,	2048000011};

	double tab2[12];

for(j=0 ; j<12 ; j++)
{
	deb = clock();
	
	i=2;
	prem = 1;

	while( i <= tab1[j]-1 && prem == 1){

		if( tab1[j]%i == 0)
			prem = 0;
		else
			i = i + 1;
	}
	fin = clock();

	total = (double) (fin - deb)/CLOCKS_PER_SEC;
	
	tab2[j]= total ;
	printf("%lf, ", tab2[j]);
}
return 0;

}

\end{sql}

\subsection{Représentation par un graphe, Gf1(n) et Gf2(n), les variations de la fonction de la complexité temporelle correspondant au meilleur cas f1(n) et au pire cas f2(n) en fonction de n respectivement; et par un autre graphe, noté GT(n), les variations  du temps d'exécution T(n) en fonction de n.}

\subsubsection{Représentation des deux graphes Gf1 et Gf2 du meilleur et pire cas respectivement :}
Sachant que pour le meilleur cas c'est une constante et pour le pire cas le graphe aura une forme linéaire vu sa fonction.
\\
\includegraphics[width=1\textwidth]{graphe/Pire_VS_Meilleur_cas1.png}

\subsubsection{Représentation du graphe GT de la variation du temps d'exécution selon les données de teste}

\includegraphics[width=1\textwidth]{graphe/Algorithme1.png}


\subsection{Interprétation des résultats.}
\subsubsection{Comparaison des mesures de temps d'exécution avec le pire et meilleur cas:}
\begin{enumerate}
	\item Remarque: \\
	En comparant le graphe de la variation du temps d'exécution de l'échantillon GT avec les graphes du meilleur f1 et pire f2 cas, nous constatons que GT tant à ressembler au graphe f1; ils ont tous deux une forme polynomial linéaire.
	\\
	  

	\item Signification:
	\\
	Les données de l'échantillon correspondent au pire cas, effectivement se sont tous des nombres premiers.
	

\end{enumerate}

\subsubsection{Remarque et déduction d'une fonction T(n) reliant n au temps d'exécution.}

On remarque que les temps d'exécution sont approximativement doublés lorsque N est doublé.
\\

\color{blue}
Exemples:
\color{black} 
\\
N1 = $4000037  \Rightarrow  $  T1 = 0.013000
\\
N2 = $8000009 \approx 2 * N1  \Rightarrow  $  T2 = 0.026000 $\approx 2 * T1 $
\\

Aussi
\\
N1 = $128000003 \Rightarrow $  T1 = 0.411000
\\
N2 = $256000001 \approx 2 * N1 \Rightarrow $  T2 = 0.820000 $\approx 2 * T1 $
\\

On peut constater la linéarité du graphe. 
\\
\\
On en déduit que le temps d'exécution est proportionnel à N, ce que l'on peut représenter par la formule suivante
: 
\begin{center}
\color{blue}
	$T(x*N) = x*T(N)$ pour tous $ x*N \in [1000003 - 2048000011] $	
	
\color{black}
(x étant la tangente d'un point sur le graphe).
\end{center}

Nous ne pouvant pas généraliser car les testes que nous avons fait n'englobent pas toutes les valeurs possibles, 
	



\subsubsection{Comparaison de la complexité théorique et expérimentale. }
Dans le cas des données de l'échantillon la complexité théorique et expérimentale sont du même ordre de grandeur que la complexité théorique du pire cas,
\\
Mais en générale la complexité expérimentale d'un échantillon quelconque est compris entre la complexité au pire cas et au meilleur cas.
\\
\begin{center}
\color{blue}
Complexité Meilleur cas $\le$ complexité expérimentale $\le$ complexité Pire cas 
\\
f2(n) $\le$ GT(n) $\le$ f1(n)
\color{black}
\end{center}

\section{Partie II: Algorithme 2 du test de la primalité.}

\subsection{Développement de l'algorithme qui permet de déterminer  si un nombre entier naturel n est premier (n$>$=2). }
Afin de vérifier si un nombre entier naturel "n" est premier ou pas nous allons tester s'il est divisible par un autre nombre entier naturel appartenant à l'intervalle $[2 - n/2]$.
Pour cela nous allons utiliser la fonction modulo qui donne le reste de la division de n par i, i variant de 2 jusqu'à $n/2$.


\begin{sql}

 Algorithme_nombre_premier2

 VAR
 i,N : entier;
 prem : booleen;
 
 DEBUT
 
	écrire("donner la valeur de N = ");
	lire(N);

	i = 2;
	prem = vrai;

	tant que( i <= (N div 2) ET prem == vrai){
		si( N mod i == 0)
			alors 
				prem = faux;				
			sinon
				i = i + 1;
	}

	si(prem == 1)
    	alors
        	écrire("Le nombre saisi :",N,"est premier!");
    	sinon
        	écrire("Le nombre saisi :",N,"n'est pas premier! ");
 FIN. 
\end{sql}

\subsection{Complexité:}

\subsubsection{Calcule des complexités temporelles en notation exacte et/ou en notation asymptotique de Landau O (Grand O) de  cet  algorithme au meilleur cas, notée f1(n), et au pire cas, notée f2(n). }
Le calcule de la complexité exact de cet algorithme n'est pas possible car nous ne pouvons pas déterminer une forme générale représentant les nombres premiers.
 
\begin{enumerate}
	\item Calcule de la complexité au meilleur cas:
	\\
	Il s'agit du cas ou le nombre est égale à 2 donc la boucle n'est exécutée aucune fois, tel que :
	\\
	f1(n) = 1(=) + 1(=) + 4($<$= , div , == , et) 
	\\
	\color{blue}
	f1(n) = 6 (Opérations) $\Rightarrow$ f1(n) = O(1)
	\color{black}
	\\
	\item Calcule de la complexité au pire cas:
	\\
	Il s'agit du cas ou le nombre est premier c'est à dire le contenu de la boucle est exécuté $\lfloor{\frac{n}{2}}$ - 2 + 1 fois = $\lfloor{\frac{n}{2}}$ - 1 fois.
	(selon la règle : fin-debut+1)
	\\
	ce qui donne:
	\\
	f2(n) = 1(=) + 1(=) + 4($<$=,div,==,et)($\lfloor{\frac{n}{2}}$) + [2(mod,==)+2(+,=)]($\lfloor{\frac{n}{2}}$-1) 
	\\
	\color{blue}
	f2(n) = 8$\lfloor{\frac{n}{2}}$ - 2 (Opérations) $\Rightarrow$ f2(n) = O($\lfloor{\frac{n}{2}}$) = O(n)
	\color{black}
\end{enumerate}




\subsubsection{Calculer la complexité spatiale en notation exacte et/ou en notation asymptotique de Landau O (Grand O) de  cet  algorithme notée s(n).}
Il s'agit du nombre de case mémoire ou octets utilisés par le programme.
\\
Pour calculer la complexité spatiale de cet algorithme nous allons considérer les tailles mémoires des types en langage C; tel que nous aurons:
int/entier : 2 octets
\\
\begin{enumerate}
	\item En notation exacte:
	\\
Nous avons dans cet algorithme que trois (3) variables de type "entier"
\\
Ce qui nous fait :
\color{blue}
 3(2 octets) = 6 octets
\color{black}
\\
 
	\item En notation asymptotique:
	\\
	Le nombre de case mémoire est constant, donc on obtient:
	\\
	\color{blue}
	s(n) = O(1)
	\color{black}
	
	
\end{enumerate}


\subsection{Développement de programme correspondant avec le langage C.}


\begin{sql}
#include <stdlib.h>
#include <stdio.h>

int main()
{
	int i,N,prem;

	printf("donner la valeur de N = ");
	scanf("%d",&N);

	i=2;
	prem = 1;

	while( i <= N/2 && prem == 1){
		if( N%i == 0)
			prem = 0;
		else
			i = i + 1;
	}

	if(prem == 1)
    {
        printf("Le nombre saisi : %d est premier! \n",N);
    }
	else{
        printf("Le nombre saisi : %d n'est pas premier! \n",N);
	}
return 0;
}
\end{sql}



\subsection{Vérification par programme  si  les  nombres  n  donnés  dans  le tableau de l'énoncé (1.000.003, 2.000.003, …) sont premiers.}

Pour répondre à cette question, notre programme doit contenir une tableau tab[ ] des valeurs à tester, ainsi le programme de teste de primalité devra être exécuté pour chaque élément du tableau.

\subsubsection{Programme C à exécuter:}

\begin{sql}
#include <stdlib.h>
#include <stdio.h>
#include <time.h>

int main()
{
	long int i,j,prem;

	long int tab[]={1000003, 2000003, 4000037, 8000009, 16000057, 32000011, 64000031, 128000003, 256000001, 512000009,	1024000009,	2048000011};

for(j=0 ; j<12 ; j++)
{
	i=2;
	prem = 1;
	
	while( i <= tab[j]/2 && prem == 1){
		if( tab[j]%i == 0)
			prem = 0;
		else
			i = i + 1;
	}

	if(prem == 1)
    {
        printf("Le nombre saisie : %ld est premier! \n",tab[j]);
    }
	else{
        printf("Le nombre saisie : %ld n'est pas premier! \n",tab[j]);
	}
}
return 0;
}
\end{sql}

\subsubsection{Résultats de l'exécution du programme:}
\begin{sql}
Le nombre saisie : 1000003 est premier!
Le nombre saisie : 2000003 est premier!
Le nombre saisie : 4000037 est premier!
Le nombre saisie : 8000009 est premier!
Le nombre saisie : 16000057 est premier!
Le nombre saisie : 32000011 est premier!
Le nombre saisie : 64000031 est premier!
Le nombre saisie : 128000003 est premier!
Le nombre saisie : 256000001 est premier!
Le nombre saisie : 512000009 est premier!
Le nombre saisie : 1024000009 est premier!
Le nombre saisie : 2048000011 est premier!
\end{sql}

On remarque que tous les nombres données sont premiers!

\subsection{Mesure des temps d'exécution T pour les nombres n données.}

Pour mesurer le temps d'exécution du programme nous utilisons les fonctions de gestion du temps qui sont fournies dans la bibliothèque "time.h" .

\subsubsection{Programme C correspondant au calcule du temps d'exécution pour chaque valeur du tableau:}
\begin{sql}
#include <stdlib.h>
#include <stdio.h>
#include <time.h>

int main()
{
	long int i,j,prem;
	clock_t deb,fin;
	double total;

	long int tab[]={1000003, 2000003, 4000037, 8000009, 16000057, 32000011,	64000031, 128000003, 256000001,	512000009,	1024000009, 2048000011};

for(j=0 ; j<12 ; j++)
{
	deb = clock();
	
	i=2;
	prem = 1;

	while( i <= tab[j]/2 && prem == 1){
		if( tab[j]%i == 0)
			prem = 0;
		else
			i = i + 1;
	}

	fin = clock();

	if(prem == 1)
    {
        printf("Le nombre saisie : %ld est premier! \n",tab[j]);
    }
	else{
        printf("Le nombre saisie : %ld n'est pas premier! \n",tab[j]);
	}

	total = (double) (fin - deb)/CLOCKS_PER_SEC;
	printf("temps d'exécution = %lf \n",total);
}
return 0;
}
\end{sql}

\subsubsection{Résultat de l'exécution du programme:}
\begin{sql}
Le nombre saisie : 1000003 est premier!
temps d`exécution = 0.004000
Le nombre saisie : 2000003 est premier!
temps d`exécution = 0.007000
Le nombre saisie : 4000037 est premier!
temps d`exécution = 0.009000
Le nombre saisie : 8000009 est premier!
temps d`exécution = 0.016000
Le nombre saisie : 16000057 est premier!
temps d`exécution = 0.026000
Le nombre saisie : 32000011 est premier!
temps d`exécution = 0.051000
Le nombre saisie : 64000031 est premier!
temps d`exécution = 0.102000
Le nombre saisie : 128000003 est premier!
temps d`exécution = 0.205000
Le nombre saisie : 256000001 est premier!
temps d`exécution = 0.415000
Le nombre saisie : 512000009 est premier!
temps d`exécution = 0.821000
Le nombre saisie : 1024000009 est premier!
temps d`exécution = 1.647000
Le nombre saisie : 2048000011 est premier!
temps d`exécution = 3.284000

\end{sql}


\subsubsection{Remplissage du tableau:}
\color{blue}
\textrm{  }
\\
\\
\begin{tabular}{|p{3cm}||p{1.8cm}|p{1.8cm}|p{1.8cm}|p{1.8cm}|p{1.8cm}|p{1.8cm}|}
\hline
Valeur N : & 1000003 & 2000003 & 4000037 & 8000009 & 16000057  & 32000011\\
\hline
Temps d'exe : & 0.004000 & 0.007000 & 0.009000 & 0.016000 & 0.026000 & 0.051000 \\
\hline
\end{tabular}
\\
\\
\begin{tabular}{|p{3cm}||p{1.8cm}|p{1.8cm}|p{1.8cm}|p{1.8cm}|p{1.8cm}|p{1.8cm}|}
\hline
Valeur N : & 64000031 & 128000003 & 256000001 & 512000009 &  1024000009 & 2048000011\\
\hline
Temps d'exe : &  0.102000 & 0.205000 & 0.415000 & 0.821000 &  1.647000 & 3.284000 \\
\hline
\end{tabular}


\textrm{  }
\\
\color{black}

\subsection{Développement du programme de mesure du temps d'exécution du programme qui a en entrée les données de l'échantillon dans tab1 et en sortie les temps d'exécution dans tab2. }
\begin{sql}
#include <stdlib.h>
#include <stdio.h>
#include <time.h>

int main()
{
	long int i,j,prem;
	clock_t deb,fin;
	double total;

	long int tab1[]={1000003, 2000003,	4000037,	8000009,	16000057,	32000011,	64000031,
	128000003,	256000001,	512000009,	1024000009,	2048000011};

	double tab2[12];

for(j=0 ; j<12 ; j++)
{
	deb = clock();
	
	i=2;
	prem = 1;

	while( i <= tab1[j]/2 && prem == 1){
		if( tab1[j]%i == 0)
			prem = 0;
		else
			i = i + 1;
	}
	fin = clock();

	total = (double) (fin - deb)/CLOCKS_PER_SEC;
	
	tab2[j]= total ;
	printf("%lf, ", tab2[j]);
}
return 0;
}
\end{sql}

\subsection{Représentation par un graphe, Gf1(n) et Gf2(n), les variations de la fonction de la complexité temporelle correspondant au meilleur cas f1(n) et au pire cas f2(n) en fonction de n respectivement; et par un autre graphe, noté GT(n), les variations  du temps d'exécution T(n) en fonction de n.}


\subsubsection{Représentation des deux graphes Gf1 et Gf2 du meilleur et pire cas respectivement :}
Sachant que pour le meilleur cas c'est une constante et pour le pire cas le graphe aura une forme linéaire vu sa fonction.
\\
\includegraphics[width=1\textwidth]{graphe/Pire_VS_Meilleur_cas2.png}

\subsubsection{Représentation du graphe GT de la variation du temps d'exécution selon les données de teste}

\includegraphics[width=1\textwidth]{graphe/Algorithme2.png}

\subsection{Interprétation des résultats.}
\subsubsection{Comparaison des mesures de temps d'exécution avec le pire et meilleur cas:}
\begin{enumerate}
	\item Remarque: \\
	En comparant le graphe de la variation du temps d'exécution de l'échantillon GT avec les graphes du meilleur f1 et pire f2 cas, nous constatons que GT tant à ressembler au graphe f1; ils ont tous deux une forme polynomial linéaire.
	\\
	  

	\item Signification:
	\\
	Les données de l'échantillon correspondent au pire cas, effectivement se sont tous des nombres premiers.
	

\end{enumerate}

\subsubsection{Remarque et déduction d'une fonction T(n) reliant n au temps d'exécution.}


On remarque que les temps d'exécution sont approximativement doublés lorsque N est doublé.
\\

\color{blue}
Exemples:
\color{black} 
\\
N1 = $4000037  \Rightarrow  $  T1 = 0.009000
\\
N2 = $8000009 \approx 2 * N1  \Rightarrow  $  T2 = 0.016000 $\approx 2 * T1 $
\\

Aussi
\\
N1 = $128000003 \Rightarrow $  T1 = 0.205000
\\
N2 = $256000001 \approx 2 * N1 \Rightarrow $  T2 = 0.415000 $\approx 2 * T1 $
\\

On peut constater la linéarité du graphe. 
\\
\\
On en déduit que le temps d'exécution est proportionnel à N, ce que l'on peut représenter par la formule suivante
: 
\begin{center}
\color{blue}
	$T(x*N) = x*T(N)$ pour tous $ x*N \in [1000003 - 2048000011] $	
	
\color{black}
(x étant la tangente d'un point sur le graphe).
\end{center}

Nous ne pouvant pas généraliser car les testes que nous avons fait n'englobent pas toutes les valeurs possibles, 
	



\subsubsection{Comparaison de la complexité théorique et expérimentale. }

Dans le cas des données de l'échantillon la complexité théorique et expérimentale sont du même ordre de grandeur que la complexité théorique du pire cas,
\\
Mais en générale la complexité expérimentale d'un échantillon quelconque est compris entre la complexité au pire cas et au meilleur cas.
\\
\begin{center}
\color{blue}
Complexité Meilleur cas $\le$ complexité expérimentale $\le$ complexité Pire cas 
\\
f2(n) $\le$ GT(n) $\le$ f1(n)
\color{black}
\end{center}


\subsection{Comparaison des deux algorithmes précédents.}
\subsubsection{Représentation des deux graphes:}

\includegraphics[width=1\textwidth]{graphe/Algorithme1_VS_Algorithme2.png}

\subsubsection{Choix de l'algorithme le plus performant:}
Il est clairement visible à travers la représentation des deux graphes ci-dessus que l'algorithme2 ($\frac{n}{2}$) est plus rapide (temps d'exécution) que l'algorithme1 (n).

Donc en terme de rapidité nous choisissons l'Algorithme 2.
(vu qu'en terme d'espace mémoire ils sont équivalent)

\section{Partie III: Algorithme 3 du test de la primalité.}

\subsection{Développement de l'algorithme qui permet de déterminer  si un nombre entier naturel n est premier (n$>$=2). }
Afin de vérifier si un nombre entier naturel "n" est premier ou pas il nous allons tester s'il est divisible par un autre nombre entier naturel appartenant à l'intervalle $[2 - \sqrt(n)]$.
Pour cela nous allons utiliser la fonction sqrt qui donne la racine carré de i, i variant de 2 jusqu'à $\sqrt(n)$.


\begin{sql}

 Algorithme_nombre_premier3

 VAR
 i,N : entier;
 prem : booleen;
 
 DEBUT
 
	écrire("donner la valeur de N = ");
	lire(N);

	i=2;
	prem = vrai;

	tant que( i <= sqrt(N) ET prem == vrai){
		si( N mod i == 0)
			alors 
				prem = faux;				
			sinon
				i = i + 1;
	}

	si(prem == 1)
    	alors
        	écrire("Le nombre saisi :",N,"est premier!");
    	sinon
        	écrire("Le nombre saisi :",N,"n'est pas premier! ");
 FIN. 
\end{sql}

\subsection{Complexité:}

\subsubsection{Calcule des complexités temporelles en notation exacte et/ou en notation asymptotique de Landau O (Grand O) de  cet  algorithme au meilleur cas, notée f1(n), et au pire cas, notée f2(n). }

Le calcule de la complexité exact de cet algorithme n'est pas possible car nous ne pouvons pas déterminer une forme générale représentant les nombres premiers.
 
\begin{enumerate}
	\item Calcule de la complexité au meilleur cas:
	\\
	Il s'agit du cas ou le nombre est égale à 2 donc la boucle n'est exécutée aucune fois, tel que :
	\\
	f1(n) = 1(=) + 1(=) + 4($<$= , sqrt , == , et) 
	\\
	\color{blue}
	f1(n) = 6 (Opérations) $\Rightarrow$ f1(n) = O(1)
	\color{black}
	\\
	\item Calcule de la complexité au pire cas:
	\\
	Il s'agit du cas ou le nombre est premier c'est à dire le contenu de la boucle est exécuté $\lfloor{\sqrt{n}}$ - 2 + 1 fois = $\lfloor{\sqrt{n}}$ - 1 fois.
	(selon la règle : fin-debut+1)
	\\
	ce qui donne:
	\\
	f2(n) = 1(=) + 1(=) + 4($<$=,sqrt,==,et)($\lfloor{\sqrt{n}}$) + [2(mod,==)+2(+,=)]($\lfloor{\sqrt{n}}$-1) 
	\\
	\color{blue}
	f2(n) = 8$\lfloor{\sqrt{n}}$ - 2 (Opérations) $\Rightarrow$ f2(n) = O($\lfloor{\sqrt{n}}$)= O($\sqrt{n}$)
	\color{black}
\end{enumerate}




\subsubsection{Calculer la complexité spatiale en notation exacte et/ou en notation asymptotique de Landau O (Grand O) de  cet  algorithme notée s(n).}
Il s'agit du nombre de case mémoire ou octets utilisés par le programme.
\\
Pour calculer la complexité spatiale de cet algorithme nous allons considérer les tailles mémoires des types en langage C; tel que nous aurons:
int/entier : 2 octets
\\
\begin{enumerate}
	\item En notation exacte:
	\\
Nous avons dans cet algorithme que trois (3) variables de type "entier"
\\
Ce qui nous fait :
\color{blue}
 3(2 octets) = 6 octets
\color{black}
\\
 
	\item En notation asymptotique:
	\\
	Le nombre de case mémoire est constant, donc on obtient:
	\\
	\color{blue}
	s(n) = O(1)
	\color{black}
	
	
\end{enumerate}


\subsection{Développement de programme correspondant avec le langage C.}


\begin{sql}
#include <stdlib.h>
#include <stdio.h>
#include <math.h>

int main()
{
	int i,N,prem;

	printf("donner la valeur de N = ");
	scanf("%d",&N);

	i=2;
	prem = 1;

	while( i <= sqrt(N) && prem == 1){
		if( N%i == 0)
			prem = 0;
		else
			i = i + 1;
	}

	if(prem == 1)
    {
        printf("Le nombre saisi : %d est premier! \n",N);
    }
	else{
        printf("Le nombre saisi : %d n'est pas premier! \n",N);
	}
return 0;
}
\end{sql}



\subsection{Vérification par programme  si  les  nombres  n  donnés  dans  le tableau de l'énoncé (1.000.003, 2.000.003, …) sont premiers.}

Pour répondre à cette question, notre programme doit contenir une tableau tab[] des valeurs à tester, ainsi le programme de teste de primalité devra être exécuté pour chaque élément du tableau.

\subsubsection{Programme C à exécuter:}

\begin{sql}
#include <stdlib.h>
#include <stdio.h>
#include <time.h>

int main()
{
	long int i,j,prem;

	long int tab[]={1000003, 2000003, 4000037, 8000009, 16000057, 32000011, 64000031, 128000003, 256000001, 512000009,	1024000009,	2048000011};

for(j=0 ; j<12 ; j++)
{
	i=2;
	prem = 1;
	
	while( i <= sqrt(tab[j]) && prem == 1){
		if( tab[j]%i == 0)
			prem = 0;
		else
			i = i + 1;
	}

	if(prem == 1)
    {
        printf("Le nombre saisie : %ld est premier! \n",tab[j]);
    }
	else{
        printf("Le nombre saisie : %ld n'est pas premier! \n",tab[j]);
	}
}
return 0;
}
\end{sql}

\subsubsection{Résultats de l'exécution du programme:}
\begin{sql}
Le nombre saisie : 1000003 est premier!
Le nombre saisie : 2000003 est premier!
Le nombre saisie : 4000037 est premier!
Le nombre saisie : 8000009 est premier!
Le nombre saisie : 16000057 est premier!
Le nombre saisie : 32000011 est premier!
Le nombre saisie : 64000031 est premier!
Le nombre saisie : 128000003 est premier!
Le nombre saisie : 256000001 est premier!
Le nombre saisie : 512000009 est premier!
Le nombre saisie : 1024000009 est premier!
Le nombre saisie : 2048000011 est premier!
\end{sql}

On remarque que tous les nombres données sont premiers!

\subsection{Mesure des temps d'exécution T pour les nombres n données.}

Pour mesurer le temps d'exécution du programme nous utilisons les fonctions de gestion du temps qui sont fournies dans la bibliothèque "time.h" .

\subsubsection{Programme C correspondant au calcule du temps d'exécution pour chaque valeur du tableau:}
\begin{sql}
#include <stdlib.h>
#include <stdio.h>
#include <time.h>
#include <math.h>

int main()
{
	long int i,j,prem;
	clock_t deb,fin;
	double total;

	long int tab[]={1000003, 2000003, 4000037, 8000009, 16000057, 32000011,	64000031, 128000003, 256000001,	512000009,	1024000009, 2048000011};

for(j=0 ; j<12 ; j++)
{
	deb = clock();
	i=2;
	prem = 1;

	while( i < tab[j] && prem == 1){
		if( tab[j]%i == 0)
			prem = 0;
		else
			i = i + 1;
	}

	fin = clock();

	if(prem == 1)
    {
        printf("Le nombre saisie : %ld est premier! \n",tab[j]);
    }
	else{
        printf("Le nombre saisie : %ld n'est pas premier! \n",tab[j]);
	}

	total = (double) (fin - deb)/CLOCKS_PER_SEC;
	printf("temps d'exécution = %lf \n",total);
}
return 0;
}
\end{sql}

\subsubsection{Résultat de l'exécution du programme:}
\begin{sql}
Le nombre saisie : 1000003 est premier!
temps d`exécution = 0.000000
Le nombre saisie : 2000003 est premier!
temps d`exécution = 0.000000
Le nombre saisie : 4000037 est premier!
temps d`exécution = 0.000000
Le nombre saisie : 8000009 est premier!
temps d`exécution = 0.000000
Le nombre saisie : 16000057 est premier!
temps d`exécution = 0.000000
Le nombre saisie : 32000011 est premier!
temps d`exécution = 0.001000
Le nombre saisie : 64000031 est premier!
temps d`exécution = 0.001000
Le nombre saisie : 128000003 est premier!
temps d`exécution = 0.001000
Le nombre saisie : 256000001 est premier!
temps d`exécution = 0.002000
Le nombre saisie : 512000009 est premier!
temps d`exécution = 0.002000
Le nombre saisie : 1024000009 est premier!
temps d`exécution = 0.002000
Le nombre saisie : 2048000011 est premier!
temps d`exécution = 0.003000
\end{sql}


\subsubsection{Remplissage du tableau:}
\color{blue}
\textrm{  }
\\
\\
\begin{tabular}{|p{3cm}||p{1.8cm}|p{1.8cm}|p{1.8cm}|p{1.8cm}|p{1.8cm}|p{1.8cm}|}
\hline
Valeur N : & 1000003 & 2000003 & 4000037 & 8000009 & 16000057  & 32000011\\
\hline
Temps d'exe : & 0.000000 & 0.000000 & 0.000000 & 0.000000 & 0.000000 & 0.001000 \\
\hline
\end{tabular}
\\
\\
\begin{tabular}{|p{3cm}||p{1.8cm}|p{1.8cm}|p{1.8cm}|p{1.8cm}|p{1.8cm}|p{1.8cm}|}
\hline
Valeur N : & 64000031 & 128000003 & 256000001 & 512000009 &  1024000009 & 2048000011\\
\hline
Temps d'exe : &  0.001000 & 0.001000 & 0.002000 & 0.002000 & 0.002000 & 0.003000 \\
\hline
\end{tabular}


\textrm{  }
\\
\color{black}

\subsection{Développement du programme de mesure du temps d'exécution du programme qui a en entrée les données de l'échantillon dans tab1 et en sortie les temps d'exécution dans tab2. }
\begin{sql}
#include <stdlib.h>
#include <stdio.h>
#include <time.h>

int main()
{
	long int i,j,prem;
	clock_t deb,fin;
	double total;

	long int tab1[]={1000003, 2000003,	4000037,	8000009,	16000057,	32000011,	64000031,
	128000003,	256000001,	512000009,	1024000009,	2048000011};

	double tab2[12];

for(j=0 ; j<12 ; j++)
{
	deb = clock();
	i=2;
	prem = 1;

	while( i < tab1[j] && prem == 1){
		if( tab1[j]%i == 0)
			prem = 0;
		else
			i = i + 1;
	}
	fin = clock();

	total = (double) (fin - deb)/CLOCKS_PER_SEC;
	
	tab2[j]= total ;
	printf("%lf, ", tab2[j]);
}
return 0;
}
\end{sql}

\subsection{Représentation par un graphe, Gf1(n) et Gf2(n), les variations de la fonction de la complexité temporelle correspondant au meilleur cas f1(n) et au pire cas f2(n) en fonction de n respectivement; et par un autre graphe, noté GT(n), les variations  du temps d'exécution T(n) en fonction de n.}


\subsubsection{Représentation des deux graphes Gf1 et Gf2 du meilleur et pire cas respectivement :}
Sachant que pour le meilleur cas c'est une constante et pour le pire cas le graphe aura une forme linéaire vu sa fonction.
\\
\includegraphics[width=1\textwidth]{graphe/Pire_VS_Meilleur_cas3.png}

\subsubsection{Représentation du graphe GT de la variation du temps d'exécution selon les données de teste}

\includegraphics[width=1\textwidth]{graphe/Algorithme3.png}

\subsection{Interprétation des résultats.}
\subsubsection{Comparaison des mesures de temps d'exécution avec le pire et meilleur cas:}
\begin{enumerate}
	\item Remarque: \\
	En comparant le graphe de la variation du temps d'exécution de l'échantillon GT avec les graphes du meilleur f1 et pire f2 cas, nous constatons que GT tant à ressembler au graphe f1; ils ont tous deux une forme  racinaire.
	\\
	  
	\item Signification:
	\\
	Les données de l'échantillon correspondent au pire cas, effectivement se sont tous des nombres premiers.
	

\end{enumerate}

\subsubsection{Remarque et déduction d'une fonction T(n) reliant n au temps d'exécution.}


On remarque que les temps d'exécution sont augmentés de 0.001000 lorsque N est multipliée par 8.
\\

\color{blue}
Exemples:
\color{black} 
\\
N1 = $32000011  \Rightarrow  $  T1 = 0.001000
\\
N2 = $256000001 \approx 8 * N1  \Rightarrow  $  T2 = 0.002000 $\approx T1 + 0.00100 $
\\

Aussi
\\
N1 = $256000001 \Rightarrow $  T1 = 0.002000
\\
N2 = $2048000011 \approx 8 * N1 \Rightarrow $  T2 = 0.003000 $\approx T1 + 0.001000 $
\\

On peut constater la linéarité du graphe. 
\\
\\
On en déduit que le temps d'exécution est proportionnel à N, ce que l'on peut représenter par la formule suivante
: 
\begin{center}
\color{blue}
	$T(x*N) = T(N) + y $ pour tous $ x \in [1000003 - 2048000011]$ et $y < 1 $	
	
\color{black}

\end{center}

Nous ne pouvant pas généraliser car les testes que nous avons fait n'englobent pas toutes les valeurs possibles, 
	


\subsubsection{Comparaison de la complexité théorique et expérimentale. }

Dans le cas des données de l'échantillon la complexité théorique et expérimentale sont du même ordre de grandeur que la complexité théorique du pire cas,
\\
Mais en générale la complexité expérimentale d'un échantillon quelconque est compris entre la complexité au pire cas et au meilleur cas.
\\
\begin{center}
\color{blue}
Complexité Meilleur cas $\le$ complexité expérimentale $\le$ complexité Pire cas 
\\
f2(n) $\le$ GT(n) $\le$ f1(n)
\color{black}
\end{center}


\subsection{Comparaison des trois algorithmes précédents.}
\subsubsection{Représentation des trois graphes:}

\includegraphics[width=1\textwidth]{graphe/Algorithme1_VS_Algorithme2_VS_Algorithme3.png}

\subsubsection{Choix de l'algorithme le plus performant:}
Il est clairement visible à travers la représentation des trois graphes que l'algorithme 3 ($\sqrt{n}$) est plus rapide que d'algorithme 2 ($\frac{n}{2}$) qui est lui même plus rapide que l'algorithme 1 (n).

Donc en terme de rapidité nous choisissons l'Algorithme 3.
(vu qu'en terme d'espace mémoire ils sont équivalent)
\end{document}
